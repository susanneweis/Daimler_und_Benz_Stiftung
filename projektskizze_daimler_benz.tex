
\documentclass[11pt,a4paper]{article}
\usepackage[left=2cm,right=2cm,top=2cm,bottom=2cm]{geometry}
\usepackage[utf8]{inputenc}
\usepackage[T1]{fontenc}
\usepackage[english]{babel}
\usepackage{setspace}
\usepackage{parskip}
\usepackage{enumitem}
\usepackage{hyperref}
\usepackage{mathptmx} 
\usepackage{csquotes}
\usepackage[backend=biber,style=apa]{biblatex}
\renewcommand*{\bibfont}{\small}  % Schriftgröße der Referenzen
\usepackage{soul} % us hl to highlight
\usepackage{xcolor} % Optional, falls du die Farbe ändern willst
\usepackage[most]{tcolorbox}  % in der Präambel
\usepackage{titlesec}
\usepackage{changepage}
\usepackage{comment}
\usepackage{tabularx}
\usepackage[table]{xcolor} % Optional: für dezente Hintergrundfarben
\usepackage{booktabs}      % Für schönere Tabellenlinien
\usepackage{longtable}
\usepackage{array}
\usepackage[table]{xcolor}

% Vor der Tabelle:
\renewcommand{\arraystretch}{1.2}
\rowcolors{2}{gray!10}{white}

\titlespacing*{\section}{0pt}{0.8ex plus 0.5ex minus .2ex}{0.3ex}
\titlespacing*{\subsection}{0pt}{0.8ex plus 0.5ex minus .2ex}{0.3ex}
\titlespacing*{\subsubsection}{0pt}{0.8ex plus 0.5ex minus .2ex}{0.3ex}
\titleformat{\paragraph}[block]{\normalfont\normalsize\bfseries}{\theparagraph}{1em}{}
\titlespacing*{\paragraph}{0pt}{0.5ex plus 0.2ex minus 0.1ex}{1ex}

\sethlcolor{yellow} % Setzt die Highlight-Farbe
\setlist[itemize]{leftmargin=*, topsep=-3pt, itemsep=0pt}
\setstretch{1.3}

\addbibresource{/Users/sweis/Data/Arbeit/Bibliothek/MasterBib_LaTex/MasterBib_LaTex.bib}

\begin{document}

\noindent
\section*{\Large\textbf{Dynamic Cognition: \\Using Movies to unravel Sex and Gender differences in the Brain in Action}}
\hfill

\section*{Ansprechpartnerin}
Susanne Weis, Heinrich Heine University Düsseldorf; \\
Gruppenleiterin „Variabilität des Gehirns“, Gehirn und Verhalten (INM-7), Institut für Neurowissenschaften und Medizin,  
Forschungszentrum Jülich; \\
Heilpraktikerin für Psychotherapie; \\
E-Mail: S.Weis@fz-juelich.de

\section*{Kurzzusammenfassung} 
Das zentrales Ziel des Projekts ist es, wissenschaftlich fundiertes Wissen über geschlechtsspezifische Unterschiede in Kognition, 
Emotion, Erleben und Verhalten in die psychische Gesundheitsversorgung zu integrieren und so dazu beizutragen, dass die unterschiedlichen 
Bedürfnisse von Frauen und Männern in Diagnostik und Therapie systematisch berücksichtigt werden.\\
Aktuelle wissenschaftliche Erkenntnisse belegen ausgeprägte Geschlechtsunterschiede in der strukturellen und funktionellen
Organisation des Gehirns. Diese Unterschiede spiegeln sich in kognitiven Prozessen und Verhaltensweisen wider, insbesondere im 
Erleben von und im Umgang mit Emotionen, welche eine wichtige Rolle im Bereich der psychischen Gesundheit spielen.
Vor diesem Hintergrund sind signifikante Geschlechtsunterschiede in der Prävalenz psychischer Erkrankungen nicht überraschend. 
So erkranken Frauen häufiger an Depressionen, Angststörungen und Essstörungen, während Männer häufiger von 
Suchterkrankungen, Aufmerksamkeits-defizit- / Hyperaktivitätsstörungen (ADHS) und externalisierenden 
Störungen betroffen sind.\\ 
Die für die Diagnostik und Therapie psychiatrischer Erkrankungen wichtigen Geschlechtsunterschiede beruhen jedoch 
nicht nur auf biologisch begründeten neuropsychologischen Unterschieden, sondern auch auf sozial erlernten Geschlechterrollen und gesellschaftlichen Erwartungen. Männer stehen häufig
unter dem Druck, Stärke und Kontrolle zeigen zu müssen, was zur Unterdrückung emotionaler Symptome und einer geringeren Inanspruchnahme 
psychotherapeutischer Hilfe führt. Frauen hingegen werden emotionales Verhalten und Vulnerabilität gesellschaftlich eher zugeschrieben, 
wodurch ihre Symptome schneller als behandlungsbedürftig erkannt werden. Gleichzeitig besteht jedoch die Gefahr der Überpathologisierung. 
Geschlechtsspezifische Normen beeinflussen das Selbstbild, die Wahrnehmung von Symptomen sowie die 
therapeutische Beziehung. Sie müssen daher zwingend in eine differenzierte, geschlechtersensible Diagnostik und 
Behandlung einbezogen werden.\\
Somit liegt der Fokus des hier vorgeschlagenen Projektes auf der Vermittlung neurowissenschaftlicher Erkenntnisse über 
Geschlechtsunterschiede in Gehirn und Verhalten an betroffene Patient:innen und deren Angehörige, sowie an Therapeut:innen 
mit dem Ziel, die Relevanz dieser Unterschiede für psychische Erkrankungen und deren Behandlung in den Fokus der 
Aufmerksamkeit aller Beteiligen zu bringen.\\
Wir wollen das Bewusstsein für geschlechtsbezogene Einflussfaktoren in Psychotherapie und Beratung 
schärfen und die psychische Gesundheitskompetenz in der Bevölkerung stärken. Das Projekt versteht sich somit als Brücke zwischen 
neurowissenschaftlicher Forschung und psychotherapeutischer Praxis. Es leistet damit einen konkreten Beitrag zu einer 
individualisierten und gerechten psychischen Gesundheitsversorgung, in der Geschlecht nicht als Randaspekt, 
sondern als integrativer Bestandteil von Diagnostik und Therapie verstanden wird.\\
Durch eine verständliche, praxisnahe und zugängliche Vermittlung wissenschaftlicher Erkenntnisse 
zu Geschlechtsunterschieden in Gehirn, Verhalten und Erleben wird ein differenziertes Verständnis 
psychischer Erkrankungen gefördert - jenseits von pseudowissenschaftlichen Stereotypen.
So entsteht Raum für eine zukunftsorientierte Psychotherapie, die neurobiologische, psychologische und 
gesellschaftliche Perspektiven miteinander verbindet und den Menschen in seiner ganzen Vielfalt in den 
Mittelpunkt stellt.

\section*{Ausführliche Vorhabensbeschreibung}

\subsection*{Thema und Idee}

Psychische Erkrankungen zeigen ausgeprägte Geschlechtsunterschiede in Prävalenz, Alter bei Erkrankung und Symptombild. 
Frauen sind häufiger von Depressionen, Angst- und Essstörungen betroffen, Männer von ADHS, Autismus-Spektrum- und 

\end{document}
